\documentclass[]{article}
\usepackage{lmodern}
\usepackage{amssymb,amsmath}
\usepackage{ifxetex,ifluatex}
\usepackage{fixltx2e} % provides \textsubscript
\ifnum 0\ifxetex 1\fi\ifluatex 1\fi=0 % if pdftex
  \usepackage[T1]{fontenc}
  \usepackage[utf8]{inputenc}
\else % if luatex or xelatex
  \ifxetex
    \usepackage{mathspec}
  \else
    \usepackage{fontspec}
  \fi
  \defaultfontfeatures{Ligatures=TeX,Scale=MatchLowercase}
\fi
% use upquote if available, for straight quotes in verbatim environments
\IfFileExists{upquote.sty}{\usepackage{upquote}}{}
% use microtype if available
\IfFileExists{microtype.sty}{%
\usepackage{microtype}
\UseMicrotypeSet[protrusion]{basicmath} % disable protrusion for tt fonts
}{}
\usepackage[margin=1in]{geometry}
\usepackage{hyperref}
\hypersetup{unicode=true,
            pdftitle={BINF702 Spring 2019 - Golub Revisited: Molecular Classification of Cancer Using Statistical Methods},
            pdfauthor={Alexander Smith},
            pdfborder={0 0 0},
            breaklinks=true}
\urlstyle{same}  % don't use monospace font for urls
\usepackage{graphicx,grffile}
\makeatletter
\def\maxwidth{\ifdim\Gin@nat@width>\linewidth\linewidth\else\Gin@nat@width\fi}
\def\maxheight{\ifdim\Gin@nat@height>\textheight\textheight\else\Gin@nat@height\fi}
\makeatother
% Scale images if necessary, so that they will not overflow the page
% margins by default, and it is still possible to overwrite the defaults
% using explicit options in \includegraphics[width, height, ...]{}
\setkeys{Gin}{width=\maxwidth,height=\maxheight,keepaspectratio}
\IfFileExists{parskip.sty}{%
\usepackage{parskip}
}{% else
\setlength{\parindent}{0pt}
\setlength{\parskip}{6pt plus 2pt minus 1pt}
}
\setlength{\emergencystretch}{3em}  % prevent overfull lines
\providecommand{\tightlist}{%
  \setlength{\itemsep}{0pt}\setlength{\parskip}{0pt}}
\setcounter{secnumdepth}{0}
% Redefines (sub)paragraphs to behave more like sections
\ifx\paragraph\undefined\else
\let\oldparagraph\paragraph
\renewcommand{\paragraph}[1]{\oldparagraph{#1}\mbox{}}
\fi
\ifx\subparagraph\undefined\else
\let\oldsubparagraph\subparagraph
\renewcommand{\subparagraph}[1]{\oldsubparagraph{#1}\mbox{}}
\fi

%%% Use protect on footnotes to avoid problems with footnotes in titles
\let\rmarkdownfootnote\footnote%
\def\footnote{\protect\rmarkdownfootnote}

%%% Change title format to be more compact
\usepackage{titling}

% Create subtitle command for use in maketitle
\providecommand{\subtitle}[1]{
  \posttitle{
    \begin{center}\large#1\end{center}
    }
}

\setlength{\droptitle}{-2em}

  \title{BINF702 Spring 2019 - Golub Revisited: Molecular Classification of
Cancer Using Statistical Methods}
    \pretitle{\vspace{\droptitle}\centering\huge}
  \posttitle{\par}
    \author{Alexander Smith}
    \preauthor{\centering\large\emph}
  \postauthor{\par}
      \predate{\centering\large\emph}
  \postdate{\par}
    \date{10/24/2019}

\usepackage{setspace}
\doublespacing

\begin{document}
\maketitle

\section{Abstract}\label{abstract}

The accuracy of cancer classification is of the utmost importance to the
diagnosis and effective treatment of any form of cancer, both for
maximizing the efficacy and for minimizing the toxicity of treatment.
The Golub 1999 study explored methods of improving the accuracy of acute
lymphoblastic leukemia (ALL) and acute myeloid leukemia (AML)
classification and diagnosis using through statistical methods applied
to microarray gene expression data acquired through Affymetrix gene chip
technology {[}@golub{]}. The Golub study was groundbreaking in what it
achieved, and its methods were a novel application of statistical
learning techniques which are still applied today. The goal of this
study is to replicate the methods and techniques used in the Golub study
using tools and technology in the R programming language and
Bioconductor ecosystem. In doing so, we will validate the techniques
used in the original Golub 1999 study as well as potentially discover
new techniques through the application of modern methods.

\section{Background and Objectives}\label{background-and-objectives}

The Golub 1999 study sought to discover statistical learning methods
which could accurately distinguish and classify molecular subtypes of
leukemia, specifically acute lymphoblastic leukemia (ALL) and acute
myeloid leukemia (AML). The study divided cancer classification into two
main challenges: class discovery and class prediction. Those challenges
map to unsupervised and supervised learning techniques, respectively.

Through replication of the Golub data analysis, we aim to answer
questions related to leukemia subtype classification through using newer
tools and techniques, such as statistical analysis and machine learning.
In doing so, we make no prior assumptions about our results, but we do
aim to either validate the techniques used in the original study or
apply other techniques of answering these questions which involve
machine learning methods. We also aim to elaborate on related topics,
such as techniques for validating the accuracy of our results or insight
into identifying optimal feature sets for class prediction and discovery
of leukemia subtypes. In short, this study aims to discover new
techniques used in the classification of leukemia subtypes or validate
existing techniques.

\begin{enumerate}
\def\labelenumi{\arabic{enumi}.}
\item
  What are the genes or biomarkers associated with AML and ALL
  classification? What is an optimal set for the class prediction?
\item
  What are the molecular subtypes of AML and ALL that can be accurately
  predicted? Which subtypes can be accurately identified through class
  discovery?
\item
  What are some of the advantages and disadvantages of different
  statistical learning techniques used with class predicition and class
  discovery?
\end{enumerate}

\section{Computational Methods}\label{computational-methods}

In this study, we will pursue the following in order to attempt to meet
the objectives given above:

\begin{itemize}
\item
  We will use descriptive statistical methods in R to analyze the Golub
  gene expression data in order to find appropriate biomarkers and thus
  define a feature set.
\item
  We will use statistical learning and machine learning methods in R to
  classify leukemia subtypes based on gene expression data.
\item
  We will use statistical learning and machine learning methods in R to
  discover leukemia subtypes with no prior biological knowledge based on
  gene expression data.
\item
  We will use validation techniques to verify our results and estimate
  their accuracy.
\end{itemize}

Before anything else, a good understanding of the data is needed.
Checking the mean and standard deviation will determine if the data has
already been subjected to any standardizing transformations. The
application of a normality test, such as Shapiro-Wilk, will tell us if
the data is normally distributed.

Next, the Golub data needs to be explored such that we find the
biomarkers of most interest. This step is commonly referred to as
feature selection, and in this case of primary interest are gene
expression samples which differ enough between the two groups to be
useful for classification or discovery. A relatively simple method in
which to accomplish this is to test the difference in medians across the
two patient groups. For this study's purposes, the top 50 biomarkers
should be enough.

Before attempting class prediction, the data needs to be split into a
training set and a test set so that we can later validate our results.
For this study, a randomly selected sample of 1/3rd of the data will be
set aside as a test set. The remaining 2/3rds will be the training set.
The training set is used to train the model, while the test set is used
to gauge model performance.

This study will employ a number of methods for class prediction,
including Random Forest, Support Vector Machines, and K Nearest
Neighbor. Random Forest is the first choice, as it provides both good
predictive power along with good interpretability of results through
viewing the resulting tree. In addition, using Random Forest allows us
to determine the features which had the greatest effect in making our
predictions.

We will also train a Support Vector Machine to use as a challenger
model. The use of a challenger model helps to verify our predicitive
strength. In other words, if the random forest and support vector
machine models achieve similar results, we can be confident in the
accuracy of our model.

Additionally, we will attempt class prediction using a K Nearest
Neighbor model. Essentially, this serves as a second challenger model,
but now we will have three models on which to gauge our prediction
performance.

For class discovery, we cluster the Golub data using a number of
methods, such as hierarchical and K-Means clustering. Hierarchical
clustering builds a hierarchy between observations based on their
distance from each other, providing a clear view of the cluster
hierarchies when plotted as a dendrogram. K-Means evenly distributes a
data set into a predetermined number of clusters where each data point
is clustered according to its distance from a centroid.

\section{Results Discussion}\label{results-discussion}

An initial attempt at classification using a random forest achieves
promising results, and we see that the test set was classified
completely correctly (Figure 2).

The variable importance plot of the top 10 variables gives us an idea of
some of the important genes which may be related to ALL or AML. We'll
look at the top 2 later: CST3 Cystatin C and Zyxin.

From the plot of the cluster dendrogram, we see at least 2 major groups
and perhaps other well formed clusters within them.

Applying K-Means clustering with K=2 will easily find 2 clusters which
map to ALL and AML groups. However, the previous hierarchical clusters
indicated that there may be more groups of interest than just two.
Indeed in the original Golub study, the discovery process found 3
groups.

Of course, K-Means will find any number of clusters you tell it to,
though that doesn't mean the clusters are meaningful. In the original
Golub study, class discovery was paired with class prediction in order
to validate the results of discovery. To summarize, if classification
showed high predictive strength after class discovery, then there can be
more confidence that the discovered classes are correct.

\section{Conclusions}\label{conclusions}

\section{Further Study}\label{further-study}

A brief description of how the conclusions of your analyses could be
tested using biochemical or genetic techniques

\section{References}\label{references}


\end{document}
